\documentclass[11pt,a4paper]{book}
\usepackage[utf8]{inputenc}
\usepackage[english]{babel}
\usepackage{paralist}

\usepackage{doavoidnot}

\author{Gergő Balogh}
\title{Code Convention}
\begin{document}
\maketitle

\chapter{C\#}

\section{Capitalization styles}
\paragraph{Pascal case} The first letter in the identifier and the first letter of each subsequent concatenated word are capitalized. \verb|PascalCase|
\paragraph{Camel case} The first letter of an identifier is lowercase and the first letter of each subsequent concatenated word is capitalized. \verb|camelCase|
\paragraph{Uppercase} All letters in the identifier are capitalized. \verb|UPPERCASE|

\section{Naming}

\domark Use pascal case if the visibility of entity is public.
\begin{verbatim}
public
void Something()
\end{verbatim}

\domark Use camel case if the visibility of method, attribute, property or event is non-public.
\begin{verbatim}
private
void something()

internal
void something()
\end{verbatim}

\domark Use camel case and start the identifier with an underscore (\_) if the visibility of variable is local.
\begin{verbatim}
internal
void something()
{
    int _count;
}
\end{verbatim}

\whymark You can use code completion to quickly list all local variables, just type the underscore sing (\_).

\domark Use camel case and start the name of parameters of lambda expressions with an underscore (\_).
\begin{verbatim}
var foo = ( _a, _b ) => _a + _b;
\end{verbatim}

\domark Use the variable name \_ (a single underscore) to denote variables in lambda expressions, which is irrelevant in the current context.
\begin{verbatim}
(_, _b) => _b + 1
\end{verbatim}

\domark Use pascal case in the name of namespaces, classes, interfaces, structures, enumerations and members of enumerations.

\domark Start name of interfaces with a capital I.

\domark Start name of generic type parameters with a capital T.
\begin{verbatim}
class ExampleClass< TValue >

interface IExampleInterface

enum ExampleEnum

struct ExampleStruct
\end{verbatim}

\domark Use camel case in the name of parameters of methods.
\begin{verbatim}
void Somthing( int somethingToRead )
\end{verbatim}

\notmark Do not use Hungarian notion.
\begin{verbatim}
int i_foo;
float f_foo;
\end{verbatim}

\whymark It usually results cryptic abbreviation and types are noted and shown by Visual Studio or any other IDE.

\avoidmark Avoid using single letter variable names.
\begin{verbatim}
//avoid
int a;
int b;

//do
int componentA;
int componentB;
\end{verbatim}

\notmark Do not mark member with m\_ prefix.
\begin{verbatim}
int m_length //wrong

int length //right
\end{verbatim}

\notmark Do not use white-spaces to construct columns in the source code. 
\begin{verbatim}
//do
int length = 10;
string name = "Thangorodrim;

//wrong
int     length  =   10;
string  name    =   "Thangorodrim;
\end{verbatim}

\whymark It suggest structures which do not exist.

\avoidmark Avoid uncommon abbreviation. Use abbreviation, when you have no other choice.
\begin{verbatim}
string Idkwit = "true"; //I Don't Know What Is This
\end{verbatim}

\domark Use the same case as in their first letter of abbreviation less then three character long.
\begin{verbatim}
private
void uiReader()

private
void readerUI()

public
void UIReader()
\end{verbatim}

\domark Use camel case in abbreviation more then two letters long.
\begin{verbatim}
private
void umlReader()

public
void UmlReader()
\end{verbatim}

\avoidmark Avoid the using the following words.
\begin{compactitem}
\item data
\item information
\item some
\item do
\item make
\end{compactitem}

\whymark Because they are too general to provide useful information.

\notmark Do not use plural form in the name of namespaces, classes, interfaces, structures, enumerations and members of enumerations.
\begin{verbatim}
//wrong
items
men

//right
itemCollection
manCollection
\end{verbatim}

\whymark It is easy to misread the plural and singular forms. Can not find irregular forms by searching singulars.

\section{Layout}

\avoidmark Avoid lines longer then 80 characters. Prefer shorter lines as much as possible.

\subsection{Horizontal spacing}

\notmark Do not put spaces inside empty parentheses.

\domark Use spaces at the inner side of all parentheses, except curly braces.
\begin{verbatim}
( Something< int >( _foo[ bla ] ) ) * 10
\end{verbatim}

\domark Use spaces around operators, except parentheses, point (.), increment (\verb|++|) and decrement (\verb|--|) operators.
\begin{verbatim}
int foo = ( ( x == 0 ) ? ( z + 3 ) : bar.some[ 42 ] )
foo++;
\end{verbatim}

\domark Use spaces instead of tabs at the beginning of the line.

\domark Use four spaces per each indentation level. 4 spaces = 1 tab

\notmark Do not mix tabs and spaces at the beginning of the line.

\domark Only increment the indentation level one-by-one.

\subsection{Line-breaks}

\domark Put curly braces in their own empty lines.
\begin{verbatim}
void foo()
{
    if( true )
    {
        bla++;
    }
}
\end{verbatim}

\domark Put member modifiers in a separate line.
\begin{verbatim}
public static
void Something()

private
int count;
\end{verbatim}

\domark Write getter and setter with one or no modifier in the same line with the name of auto property.
\begin{verbatim}
public
int Count{ get; private set; }
\end{verbatim}

\domark If necessary break line after equal sign (=).
\begin{verbatim}
int _count =
    x + 567567563 / 2121231 + foobar;
\end{verbatim}

\domark If necessary break line before operators, except parentheses and assignment (=).
\begin{verbatim}
string _text =
    "this is a long string need to be"
    + "broken into seperate lines";

"to write something"
    .Let().TaggedAsInformation()
    .Write();
\end{verbatim}

\domark Put the closing parentheses in the same line with the last parameter, except curly braces.
\begin{verbatim}
int grantPermissions( 
    string user,
    string password,
    int id,
    PersmissionSet permission )
\end{verbatim}

\domark If necessary prefer to start method calls in a new line.

\domark Put all or none of the parameters into separate lines.
\begin{verbatim}
int count =
    AvarageOf(
        foo,
        bar,
        asd );
\end{verbatim}

\subsection{Vertical spaces}

\domark Separate members in classes with an empty lines.
\begin{verbatim}
private
int foo

public
void Something()
\end{verbatim}

\section{Commenting}

\domark Prefer documenting comments (///) over general one (//).

\notmark Do not use block comments (/* */).

\appendix

\chapter{C\#}

\section{EditorConfig for Visual Studio}

For further details and \emph{some} automatic style and convention settings please use the \texttt{.editorcongif} file next to this document. It only contains Visual Studio and C\#{} specific settings. This file do not cover every details present in this documents and provided \emph{as is}.

\end{document}